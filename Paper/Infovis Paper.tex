\documentclass{sigchi}
% Use this command to override the default ACM copyright statement
% (e.g. for preprints).  Consult the conference website for the
% camera-ready copyright statement.

%% HOW TO OVERRIDE THE DEFAULT COPYRIGHT STRIP --
%% Please note you need to make sure the copy for your specific
%% license is used here!
% \toappear{
% Permission to make digital or hard copies of all or part of this work
% for personal or classroom use is granted without fee provided that
% copies are not made or distributed for profit or commercial advantage
% and that copies bear this notice and the full citation on the first
% page. Copyrights for components of this work owned by others than ACM
% must be honored. Abstracting with credit is permitted. To copy
% otherwise, or republish, to post on servers or to redistribute to
% lists, requires prior specific permission and/or a fee. Request
% permissions from \href{mailto:Permissions@acm.org}{Permissions@acm.org}. \\
% \emph{CHI '16},  May 07--12, 2016, San Jose, CA, USA \\
% ACM xxx-x-xxxx-xxxx-x/xx/xx\ldots \$15.00 \\
% DOI: \url{http://dx.doi.org/xx.xxxx/xxxxxxx.xxxxxxx}
% }
\toappear{}

% Arabic page numbers for submission.  Remove this line to eliminate
% page numbers for the camera ready copy
% \pagenumbering{arabic}

% Load basic packages
\usepackage{balance}       % to better equalize the last page
\usepackage{graphics}      % for EPS, load graphicx instead 
\usepackage[T1]{fontenc}   % for umlauts and other diaeresis
\usepackage{txfonts}
\usepackage{mathptmx}
\usepackage[pdflang={en-US},pdftex]{hyperref}
\usepackage{color}
\usepackage{booktabs}
\usepackage{textcomp}

% Some optional stuff you might like/need.
\usepackage{microtype}        % Improved Tracking and Kerning
% \usepackage[all]{hypcap}    % Fixes bug in hyperref caption linking
\usepackage{ccicons}          % Cite your images correctly!
% \usepackage[utf8]{inputenc} % for a UTF8 editor only

% If you want to use todo notes, marginpars etc. during creation of
% your draft document, you have to enable the "chi_draft" option for
% the document class. To do this, change the very first line to:
% "\documentclass[chi_draft]{sigchi}". You can then place todo notes
% by using the "\todo{...}"  command. Make sure to disable the draft
% option again before submitting your final document.
\usepackage{todonotes}

% Paper metadata (use plain text, for PDF inclusion and later
% re-using, if desired).  Use \emtpyauthor when submitting for review
% so you remain anonymous.
\def\plaintitle{Visualizing NBA teams across multiple seasons}
\def\plainauthor[#1]{#1}
\def\plainauthors[#1]#2#3{#1, #2, #3}
\def\emptyauthor{}
\def\plainkeywords{Authors' choice; of terms; separated; by
  semicolons; include commas, within terms only; required.}
\def\plaingeneralterms{Documentation, Standardization}

% llt: Define a global style for URLs, rather that the default one
\makeatletter
\def\url@leostyle{%
  \@ifundefined{selectfont}{
    \def\UrlFont{\sf}
  }{
    \def\UrlFont{\small\bf\ttfamily}
  }}
\makeatother
\urlstyle{leo}

% To make various LaTeX processors do the right thing with page size.
\def\pprw{8.5in}
\def\pprh{11in}
\special{papersize=\pprw,\pprh}
\setlength{\paperwidth}{\pprw}
\setlength{\paperheight}{\pprh}
\setlength{\pdfpagewidth}{\pprw}
\setlength{\pdfpageheight}{\pprh}

% Make sure hyperref comes last of your loaded packages, to give it a
% fighting chance of not being over-written, since its job is to
% redefine many LaTeX commands.
\definecolor{linkColor}{RGB}{6,125,233}
\hypersetup{%
  pdftitle={\plaintitle},
% Use \plainauthor for final version.
  pdfauthor={\plainauthors[Yannick Boesmans]{Yannick Laevaert}{Yves Langeraert}},
%  pdfauthor={\emptyauthor},
  pdfkeywords={\plainkeywords},
  pdfdisplaydoctitle=true, % For Accessibility
  bookmarksnumbered,
  pdfstartview={FitH},
  colorlinks,
  citecolor=black,
  filecolor=black,
  linkcolor=black,
  urlcolor=linkColor,
  breaklinks=true,
  hypertexnames=false
}

% create a shortcut to typeset table headings
% \newcommand\tabhead[1]{\small\textbf{#1}}

% End of preamble. Here it comes the document.
\begin{document}

\title{\plaintitle}

\numberofauthors{3}
\author{%
  \alignauthor{\plainauthor[Yannick Boesmans]\\
    \affaddr{KU Leuven}\\
    \email{yannick.boesmans}\\
    \email{@student.kuleuven.be}}\\
  \alignauthor{\plainauthor[Yannick Laevaert]\\
    \affaddr{KU Leuven}\\
    \email{yannick.laevaert}\\
    \email{@student.kuleuven.be}}\\
  \alignauthor{\plainauthor[Yves Langeraert]\\
    \affaddr{KU Leuven}\\
    \email{yves.langeraert}\\
    \email{@student.kuleuven.be}}\\
}

\maketitle

\begin{abstract}
    This paper details the conceptualization and implementation of an
    interactive visualization of data about NBA seasons including teams, game
    results and player statistics. The d3.js framework was used to develop this
    visualization.
    %TODO add results
\end{abstract}

%TODO keywords
\category{H.5.m.}{Information Interfaces and Presentation
  (e.g. HCI)}{Miscellaneous} \category{See
  \url{http://acm.org/about/class/1998/} for the full list of ACM
  classifiers. This section is required.}{}{}

\keywords{\plainkeywords}

\section{Introduction}

This paper describes the visualization of NBA data made by the ``The Tufters''
team for the course ``Information Visualization''. In section~\ref{sec:goal}
we describe the goal of the visualization and the target audience. In
section~\ref{sec:data} we describe the data used, including its origins,
advantages and limitations. In section~\ref{sec:literature} we give an overview
of related literature and web resources, including related visualizations.
Related visualizations include both visualizations of NBA or other sports data,
as well as visualizations tackling a problem we encountered during the
development of our visualization. In section~\ref{sec:visualization} we describe
the visualization itself. We give an overview of the different stages of
development of the visualization, as well as the major design decisions made
during the development process. Section~\ref{sec:discussion} discusses potential
improvements of the final visualization and lessons learned from the project. We
conclude in section~\ref{sec:conclusion} by giving a short overview.

\section{Goal and Target Audience}\ref{sec:goal}
The visualization's goal is allowing exploration of NBA data by lay persons.
More specifically, the visualization does not try to offer premade explanations
for phenomenons visible in the NBA data. By providing easy and intuitive access
to the data, users can draw their own conclusions. It thus enables users of all 
sorts to explore the data for patterns.
Secondly, the visualization's target audience are lay persons. This means it 
satifies needs of people regardless if they are professionally active in the 
field of basketball or not. Specifically, fans of NBA are the core of our target
audience. This means the visualization assumes most common basketball terms are
known to the audience and as such does not provide additional information about 
them.
The visualization should enable people to look for changes in team statistics 
and evaluate the impact on the team. The visualization gives also a short insight
in what caused the change in statistics. Eg. a user could notice a spike/drop in 
the SRS score () of a team. The user will then be able to see how the spike was 
caused (team performance, players that joined the team, players that left the 
team). Next to that, the user will can also explore the impact of the change on
the team. She/he will be able to scroll through time and notice how the results of
the team change over time since the spike/drop.

\section{Data}\ref{sec:data}
The data visualized is a subset of the data available on basketball statistics
site basketball-reference\cite{basketball-reference}. A wide range of data is
available on this site. In our visualization we only use data from 1984 onwards.
The data we use includes league standings and playoff rankings for each team,
team overall statistics, the team's roster and individual player statistics 
for each year, including the PER~(Player Efficiency Rating). 

The data was gathered by scraping the basketball-reference site, mostly using
the provided download capabilities. Most of the data was downloaded in csv
format, while some tables had to be manually scraped. The data was then combined
in a preprocessing step. In this step, each team's playoff rankings were
calculated based on the matches played during the playoffs, and the rest of the
data was combined into json format. The final preprocessing step combines all
data into one json file.

\section{Related Work}\ref{sec:literature}

\section{Visualization}\ref{sec:visualization}
The visualization consists of three parts:
- The bubble view: a compact view on the play-offs per season
- The statistics (zoom) view: a more detailed view on a statistic of a selected 
team
- The team view: a detailed view on how good or bad a team scores on a specific 
field position
All three of them are discussed in more detail with the rationale behind them.

\subsection{The bubble view}
This view supposes to inform the user of the results of the NBA play-offs per 
season. A circle is used to represent a team, the size its SRS score using a ''
distribution, a stroke to illustrate it's region or medal and curved lines to
inform the user which teams played against each other. The user has a timeline
at its disposal to scroll through time.
<picture of the bubble view compared with double elimination bracket>
A more popular visualization of a play-off is the team double elimination bracket.
Although this is a more common representation, we wanted to create a more compact
version by eliminating the recurring representation of a team.
Our initial bubble view was too compact. It connected teams that played against
each other directly. When searching for alternatives to reflect on our choice we
discovered a similar visualization in a totally different context. 
<https://source.opennews.org/en-US/articles/nyts-512-paths-white-house/>
<comparison of original buble view with white house paths>
This visualization gives a cleaner view of the competitors for a specific team
compared to our original sketch. Hence we decided to adapt our visualization. We
added an intermediate step between two teams to clearer indicate how teams competed
to become NBA champion.
Next to the static view, the user can interact with the visualization. When he 
hoovers over a circle of a team, only the games played are shown by highliting the 
competitors (their circles) and the curved lines that connect them. This view gets 
fixated when the users clicks on a team.
He/she then automatically gets forwarded to the statistics view where the bubble 
view is still represented in the right upper corner to make sure the user keeps
an overview. This view will be explained in more detail in the next section.

\subsection{The statistics view}
The statistics view gives a user a clear overview of how a statistic has been 
influenced compared to the year before. The change is represented by three arrows:
- arrow on the left pointing towards the middle: indicating the influence of players
who joined the team
- arrow in the middle: indicating how the team inherent changed
- arrow on the right pointing towards the middle: indicating the influence of players
who left the team
<figure of the statistics view>
The box on the right enables the user to change between statistics. Next to that it 
informs the user of which statistic has been choosen. Below the arrows small multiples
are shown of some statistics. The line charts show how a specific team scored over
time. This enables a user to identify peaks or drops that possibly influenced the 
outcome of a team is this season/following seasons or previous seasons.
When the user hits the circle representing the teams SRS scrore, he is guided to 
the team view. This view will be explained in more detail in the next section.



\subsection{The team view}

This view as well as the statistics view should give the user the ability to search
for explanations why a team is is performing better or worse over seasons. On the 
other hand, a user can see what impact a change in team characteristics has on its
overall performance.
Initially we designed this view more elaborate. When the user would click on a shirt,
specific player information would be shown. Next to some general text information, 
we would have visualised a number of player statistics as small multiples. A histogram
should have informed how a player scores compared to his team, other players in the 
league on this position or all other players in the league.
<Sketch of the small multiples>


\subsection{Technology}
To create the visualization, the d3 javascript framework~\cite{d3} was used in
combination with html5~\cite{html5} and jquery~\cite{jquery}. This allows easy
access to the visualization as most modern browsers are capable of handling
these technologies.
The choice not to use the d3 framework for the entire visualization was made to
ease the layout configuration of the visualization. Instead, html was used to do
the global layout of the visualization.

\section{Discussion}\ref{sec:discussion}
\section{Conclusion}\ref{sec:conclusion}


% Balancing columns in a ref list is a bit of a pain because you
% either use a hack like flushend or balance, or manually insert
% a column break.  http://www.tex.ac.uk/cgi-bin/texfaq2html?label=balance
% multicols doesn't work because we're already in two-column mode,
% and flushend isn't awesome, so I choose balance.  See this
% for more info: http://cs.brown.edu/system/software/latex/doc/balance.pdf
%
% Note that in a perfect world balance wants to be in the first
% column of the last page.
%
% If balance doesn't work for you, you can remove that and
% hard-code a column break into the bbl file right before you
% submit:
%
% http://stackoverflow.com/questions/2149854/how-to-manually-equalize-columns-
% in-an-ieee-paper-if-using-bibtex
%
% Or, just remove \balance and give up on balancing the last page.
%
\balance{}

% REFERENCES FORMAT
% References must be the same font size as other body text.
\bibliographystyle{SIGCHI-Reference-Format}
\bibliography{sample}

\end{document}

%%% Local Variables:
%%% mode: latex
%%% TeX-master: t
%%% End:
